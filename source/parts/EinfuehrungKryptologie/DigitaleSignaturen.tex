\chapter{Digitale Signaturen}
\section{Anforderungen an Digitale Signaturen (9.1)}
Eine Digitale Signatur muss:
\begin{itemize}
  \item vom Nachrichteninhalt abhängen.
  \item den Absender eindeutig ausweisen.
  \item einfach zu erzeugen und zu prüfen sein.
  \item fälschungssicher sein: Der Empfänger darf keine gültige Signatur vom Sender erzeugen können. Damit kann der Sender nicht mehr leugnen die Signatur angefertigt zu haben.
  \item einfach gespeichert werden können.
  \item über einen gewissen Zeitraum beweiskräftig bleiben.
\end{itemize}

\section{Konstruktion von Digitalen Signaturen (9.2)}
Digitale Signaturen können mit asymmetrischen Kryptosystemen realisiert werden. Gegeben sei eine kryptografisch sichere Hashfunktion $H$, ein asymmetrisches Kryptosystem $(E,D)$ und dem Schlüsselpaar $(PK,SK)$. Die Entschlüsselungsfunktion $D$ wird als Signaturfunktion $sign$ benutzt. Gültige Signaturen der Nachricht $M$ des Senders sind $sign(SK, M)$ und $sign(SK, H(M))$.

Zur Prüfung wird die Verschlüsselungsfunktion $E$ als Prüffunktion $validate$ genutzt. Eine Prüfung ist genau dann erfolgreich wenn gilt:
$$
	validate(PK, sign(M)) = M
$$
beziehungsweise:
$$
	validate(PK, sign(H(M))) = H(M)
$$

Dieses Verfahren ist kryptografisch sicher, solange der Secret Key $SK$ des Senders nur dem Sender bekannt ist.

 

