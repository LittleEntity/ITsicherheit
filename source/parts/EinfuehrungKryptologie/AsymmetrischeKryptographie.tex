\chapter{Asymmetrische Kryptographie / Public Key Verfahren}
\section{Definition (5.2)}
Public Key $pK$ \\
Secret Key $sK$ \\
encrypt: $E: F(P, pK) \mapsto C $ \\
decrypt: $D: F(C, sK) \mapsto P $ \\

\begin{figure}[htp]
	\centering
	\includegraphics[scale=1]{AsymKrypto.pdf}
\end{figure}

\section{Mathematische Aufgaben auf denen asymmetrische Kryptoverfahren basieren (5.3)}

\subsection{Faktorisierung (5.3.1)}
Maschinell das Produkt zweier Primzahlen zu berechnen ist schnell. \\
Dagegen ist kein Algorithmus mit polynomieller Bekannt, welcher eine gegebene Zahl in ihre Primfaktoren zerlegt. \\
Laufzeiten von bekannten Algorithmen zur Primfaktorzerlegung einer Zahl mit $n$-bit:
\begin{itemize}
  \item Probedivision $O(2^\frac{n}{2})$
  \item Quadratisches Sieb $O(exp(n^\frac{1}{2}(ln(n))^\frac{1}{2})$
  \item Zahlenkörpersieb $O(exp(n^\frac{1}{3}(ln(n))^\frac{2}{3})$
\end{itemize}

\subsection{Diskreter Logarithmus (5.3.2)}
\begin{align*}
a &= g^e \hspace{4pt} mod \hspace{4pt} p \\
e &= log_g(a) \hspace{4pt} mod \hspace{4pt} p \\
\end{align*}
Es existieren Algorithmen $\in O(n^c)$ für ein gegebenes $e$ um die Gleichung $a = g^e \hspace{4pt} mod \hspace{4pt} p$ zu lösen.

Es sind bisher \emph{keine Algorithmen} $\in O(n^c)$ bekannt welche für gegebene $a,g,p$ die Gleichung $e = log_g(a) \hspace{4pt} mod \hspace{4pt} p$ lösen.

Aus Erfahrung wird gesagt, dass das Lösen von Diskreten-Algorithmen-Gleichungen nicht schwerer ist als das Lösen einer Faktorisierungs-Gleichung.


\subsection{Quadratwurzel ziehen mod n (5.3.3) ??????}++
"`Für n=p*q mit p und q Primzahlen gibt es ohne Kenntnis von p und q kein effizientes Verfahren zur Bestimmung der Quadratwurzeln.\\
Die Äquivalenz zum Faktorisierungsproblem kann bewiesen werden."'

\subsection{Überblick über ausgewählte Asymmetrische Verfahren (5.3.4)}
\begin{tabular}{*{5}{l}}
Verfahren & schweres Problem / Falltür & leichtes Problem & Signatur möglich & Einsatz in der Praxis \\
RSA & Faktorisierung mod n & Multiplizieren/Potenzieren mod n & ja & ja \\
Rabin & Quadratwurzel ziehen mod n & Quadrieren mod n & nein & nein \\
Elgamal & Diskreter Logarithmus mod n & Multiplizieren/Potenzieren mod n & ja & ja \\
\end{tabular}

\section{Mathematische Vorbereitung für RSA (5.4)}
\subsection{Division mit Rest}
$$
\textnormal{ganze Zahl } z \in \mathbb{Z}
$$
$$
 \textnormal{Modul } m \in \mathbb{N}
$$
$$
z = qb + r \textnormal{ mit } qb = \lfloor z : m \rfloor \textnormal{ und } r = z - qb = z \hspace{4pt} mod \hspace{4pt} m 
$$

\subsection{Modulares Rechnen}
Es gelten die gewohnten arithmetischen Rechengesetze. Jedes Teilergebnis lässt sich zusätzlich um $ mod \hspace{4pt} m$ reduzieren.

\begin{align*}
((x \hspace{4pt} mod \hspace{4pt} m) + (y \hspace{4pt} mod \hspace{4pt} m)) \hspace{4pt} mod \hspace{4pt}  m =& (x+y) \hspace{4pt} mod \hspace{4pt} m \\
((x \hspace{4pt} mod \hspace{4pt} m) \cdot (y \hspace{4pt} mod \hspace{4pt} m)) \hspace{4pt} mod \hspace{4pt}  m =& (x \cdot y) \hspace{4pt} mod \hspace{4pt} m \\
((z \hspace{4pt} mod \hspace{4pt} m) \cdot ((x \hspace{4pt} mod \hspace{4pt} m) + (y \hspace{4pt} mod \hspace{4pt} m))) \hspace{4pt} mod \hspace{4pt} m =& ((z \hspace{4pt} mod \hspace{4pt} m) \cdot (x \hspace{4pt} mod \hspace{4pt} m) \\
&+ (z \hspace{4pt} mod \hspace{4pt} m) \cdot (y \hspace{4pt} mod \hspace{4pt} m)) \hspace{4pt} mod \hspace{4pt} m \\ 
\end{align*}

\subsection{Wiederholtes Quadrieren (5.4.3)}
um $b^e \hspace{4pt} mod \hspace{4pt} n$ zu berechnen\footnote{http://de.wikipedia.org/wiki/Bin\%C3\%A4re\_Exponentiation}:
\begin{itemize}
  \item Schreibe Exponenten $e$ als Dualzahl
  \item Starte mit Zwischenergebnis $z = 1$
  \item Arbeite die Stellen vom höchstwertigsten zum niederwertigsten Bitstelle ab.
  	\begin{itemize}
  		\item wenn Stelle 0: $z := z^2$
  		\item wenn Stelle 1: $z := z^2 \cdot b$
		\end{itemize}
\end{itemize}

$$\textnormal{Beispiel: } 3^{13} \hspace{4pt} mod \hspace{4pt} 17$$
$$13_{10} = 1101_2$$
$$
\begin{array}{crclc}
	1 & 1^2 \cdot 3 & = & 3 &  (mod \hspace{4pt} 17) \\
	1 & 3^2 \cdot 3 & = & 10 &  (mod \hspace{4pt} 17) \\
	0 & 10^2 & = & -2 &  (mod \hspace{4pt} 17) \\
	1 & (-2)^2 \cdot 3 & = & 12 &  (mod \hspace{4pt} 17) \\
\end{array}
$$

\subsection{Multiplikatives Inverses (5.4.4)}
Das Multiplikatives Inverse $z^{-1}$ zu $z$ ist definiert über:
$$z \cdot z^{-1} = 1 \hspace{4pt} (mod \hspace{4pt} m) $$

\subsection{Definition: Primzahl (5.4.5)}
$p$ ist Primzahl, wenn $p$ außer sich selbst und 1 keine positiven Teiler hat.

\subsection{größter gemeinsamer Teiler (ggT) (5.4.6)}
Der ggT zweier Zahlen ist die größte Zahl, welche beide Zahlen ohne Rest teilt. \\
Falls $ggT(a,b) = 1$, so heißen a und b teilerfremd.

\subsection{Erweiterter Euklidischer Algorithmus zur Bestimmung des Multiplikativen Inversen (5.4.7)}
Wenn eine Zahl $z$ zum Modul $m$ teilerfremd ist, lässt sich ihr Multiplikatives Inverses $z^{-1}$ mit dem Erweiterten Euklidischen Algorithmus bestimmen.\footnote{genaue Erläuterung: "`http://www.yimin-ge.com/doc/multiplikative\_inverse.pdf"'} Der Algorithmus liefert auch Aufschluss über die Teilerfremdheit.
\begin{enumerate}
  \item starte mit erster Gleichung der Form $m = 1 \cdot m + 0 \cdot z$ und nachfolgender Gleichung der Form $z = 0 \cdot m + 1 \cdot z$
  \item Bilde die Folgegleichungen indem von der vorletzten Gleichung ein vielfaches der letzten Gleichung subtrahiert wird, sodass das Ergebnis der linken Seite dem Rest der Division mit Rest entspricht.
  \item Wiederhole 2. solange bis das Ergebnis der linken Seite 1 oder 0 ist.
  \begin{itemize}
    \item Wenn 1: Das Multiplikative Inverse $z^{-1}$ ist der Faktor vor $z$ .
    \item Wenn 0: Es gibt kein Multiplikatives Inverses von $z$ zum Modul $m$, da $z$ und $m$ nicht teilerfremd sind.
	\end{itemize}
\end{enumerate}
Beispiel: Finde $z^{-1}$ zu $7 \hspace{4pt} (mod \hspace{4pt} 23) $:
$$
\begin{array}{crcll}
I & 23 & = & 1 \cdot 23 + 0 \cdot 7 & \\
II & 7 & = & 0 \cdot 23 + 1 \cdot 7 & \\
III & 2 & = & 1 \cdot 23 + (-3) \cdot 7 & I + II \cdot (-3) \\
IV& 1 & = & (-3) \cdot 23 + 10 \cdot 7 & II + III \cdot (-3) \\
\end{array}
$$
Ergebnis: Das Multiplikative Inverse zu 7 zum Modul 23 ist 10.
\begin{center}
\begin{tabular}{*{4}{r}}
23 & 1 & 0 & \\
7 & 0 & 1 & -3\\
2 & 1 & -3 & -3\\
1 & -3 & \fbox{10} & \\
\end{tabular}
\end{center}

\subsection{Eulersche $\varphi$ Funktion (5.4.8)}
$\varphi$(n) = Anzahl der positiven ganzen Zahlen $z \le n$, mit $ggT(z,n) = 1$ . \\
Beispiel: $\varphi(21) = 12$ \\
speziell für Primzahlen $p, q$ : \\
$\varphi(p) = p-1$ \\
$\varphi(p \cdot q) = (p-1) \cdot (q-1)$ wobei $p \neq q$ \\

\subsection{Satz von Euler (5.4.9)}
Falls $ggT(x, n) = 1$, dann gilt  $x^{\varphi(n)} = 1 \hspace{4pt} (mod \hspace{4pt} n) $ 

\section{RSA-Verfahren (5.5)}
\subsection{Definition des RSA-Verfahrens}
Der Empfänger bildet \textsc{rsa}-Modul $n$ mit zwei Primzahlen $p$ und $q$:

$$
	n = p \cdot q
$$

Öffentliche Exponent $e$ (encrypt) wird gebildet mit:

$$
	1 < e < (p-1)(q-1) \wedge ggT(e,(p-1)(q-1)) = 1
$$

Öffentlicher Schlüssel ist das Paar $(n,e)$.

Geheime Exponent $d$ (decrypt) wird gebildet mit:

$$
	1 < d < (p-1)(q-1) \wedge e \cdot d \hspace{4pt} mod \hspace{3pt} (p-1)(q-1) = 1
$$

Geheimer Schlüssel ist Tripel $(p,q,d)$.

Hinweis: Die Berechnung von $d$ erolgt $mod \hspace{4pt} \varphi(n)$.

\subsection{RSA-Verschlüsselung und Entschlüsselung}
Encryption of Plaintext $P$ with $0 \le P < n$:
$$ C = (P^e) \hspace{4pt} (mod \hspace{4pt} n)  $$

Decryption of Cyphertext $C$:
$$ P = (C^d) \hspace{4pt} (mod \hspace{4pt} n)  $$

RSA Schlüsselänge ist Anzahl Bits des Moduls $n$.

$ P \mapsto P^e \hspace{4pt} (mod \hspace{4pt} n) $ ist eine Falltürfunktion da:
Ohne geheimem Schlüssel ist die Funktion praktisch nicht invertierbar. \\
Mit Kenntnis des geheimen Schlüssel ist die Invertierung einfach.

\subsection{Falltürfunktion (trap door function)}
 Für ein Geheimnis $S$ ist Funktion $F_S: X \mapsto Y$ eine Falltürfunktion, wenn gilt:
 \begin{enumerate}
  \item $\forall x \in X$ ist $F(x)$ leicht berechenbar.
  \item Ohne Kenntnis des Geheimnisses $S$ ist es für gegebenes $y \in Y$ praktisch nicht möglich ein Urbild $x$ von $y$ unter $F_S$ zu berechnen.
  \item Mit Kenntnis des Geheimnisses $S$ ist es für gegebenes $y \in Y$ leicht ein Urbild $x$ von $y$ unter $F_S$ zu berechnen. 
\end{enumerate}

\subsection{Sicherheit des RSA}
\begin{itemize}
  \item Aufwand um geheimen Exponenten aus öffentlichem Schlüssel zu berechnen ist äquivalent zur Faktorisierung des RSA-Moduls
  \item Bislang unbekannt RSA brechbar ohne Modul zu faktorisieren oder geheimen Exponenten $d$ zu finden.
  \item Der Aufwand zur Faktorisierung des \textsc{rsa}-Moduls hängt von dessen Größe ab.
\end{itemize}

\subsection{Sichere Auswahl von RSA Parametern}
\begin{itemize}
  \item RSA-Schlüsssellänge $\geq$ 1024 Bit.
  \item Primzahlen $p$ und $q$ sollten zufällig, gleichverteilt gewählt sein.
  \item wegen "`low exponent attack"' sollte öffentlicher Exponent $e$ nicht zu klein sein: oft wird $e = 2^{16}+1 = 65537$ gewählt.
\end{itemize}







