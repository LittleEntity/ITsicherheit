\chapter{Hashfunktionen}
\section{Anwendungen (4.1)}
\begin{itemize}
  \item Integritätscheck
  \item sichere Komprimierung von z.B. digitalen Signaturen
\end{itemize}

\section{Einwegfunktion (4.2)}
Für Einwegfunktion $F: X \mapsto Y$ gilt:
\begin{itemize}
  \item $\forall_{x \in X} : F(x) \textnormal{ leicht berechenbar}$
  \item $\forall_{y \in Y}$: es ist praktisch unmöglich ein Urbild $x \in X$ zu berechnen für das $F(x) = y $ ist.
\end{itemize}

\section{Hashfunktionen (4.3)}
Für Hashfunktion $H: \{0,1\}^* \mapsto \{0,1\}^n$
\begin{itemize}
  \item H ist Einwegfunktion
  \item Sei $x \in \{0,1\}^*$ . Es ist praktisch unmöglich ein $z \neq x$ zu berechnen, sodass $H(z) = H(x)$ gilt.
  \item Kollisionsresistenz: praktisch unmöglich frei wählbares Paar $(x,z)$ zu berechnen, sodass $H(x) = H(z)$.
\end{itemize}
Ohne die letzte Bedingung wird die Funktion schwache Hashfunktion genannt.

Umgangssprachlich: Hashfunktion ist eine kryptografisch sichere, nicht umkehrbare Komprimierungsfunktion.

\section{Bekannte Hashfunktionen (4.4)}
\begin{itemize}
  \item MD2, MD4, MD5: 128bit
  \item SHA-1: 160bit
  \item SHA-2: 224bit, 256bit, 384bit, 512bit
  \item SHA-3: 224bit bis 512bit
\end{itemize}

(4.5) SHA-1 im Detail VL2 S.12

\section{Merkle-Damgard Konstruktion von Hashfunktionen (4.6)}
Gegeben: Einwegfunktion $F: \{0,1\}^{m+n} \mapsto \{0,1\}^n$ und Initialisierungsvektor $H_0$\\
Konstruiere $F: \{0,1\}^{*} \mapsto \{0,1\}^n$
\begin{enumerate}
  \item Erweitere $P$, sodass $Länge(P)$ ein Vielfaches von $m$ ist. Das Ergebnis sei $P_m$.
  \item Teile $P_m$ in Blöcke $p_1, p_2, \ldots, p_k$ mit der Länge $m$.
  \item Berechne $H(P_m)$ mit Iterationen über die Blöcke aus 2. und der Funktion $F$ wie folgt:
  	\begin{align*}
  		H_i &= F(y_1 \circ H_0) \\
  		H_2 &= F(y_2 \circ H_1) \\
  		& \vdots \\
  		H_k &= F(y_k \circ H_{k-1}) = H(P_m) \\
		\end{align*} 
\end{enumerate}

\section{Birthday Paradox (4.7)}
Wahrscheinlichkeit P(z,N), dass mit z vielen unabhängigen Zügen aus Wertemenge \mathbb{N} ein Wert doppelt vorkommt, ist ab $z = 1,17 \cdot \sqrt{N}$ größer als $\frac{1}{2}$. 

Beispiel: Wahrscheinlichkeit dass zwei Menschen am gleichen Tag Geburtstag haben ist ab 23 Menschen größer als $\frac{1}{2}$.

\section{Anwendung des Birthday Paradox auf Hashfunktionen (4.8)}
\label{bDayHash}
Gesucht ist eine beliebige Kollision.\\
Algorithmus:
\begin{enumerate}
  \item Wähle zufällig ein bisher noch nicht betrachtetes $x_i$.
  \item Vergleiche $H(x_i)$ mit allen anderen bisher betrachteten $H(x_j)$.
  \item Wenn keine Kollision gefunden wurde, beginne wieder bei 1.
\end{enumerate} 
$H(x_i) = H(x_j)$ ergibt sich im mittel nach $ 1.17 \sqrt{2^n} $ Iterationen.

\section{Angriffe auf Hashfunktionen (4.9)}
\begin{itemize}
  \item Kollisionsangriff: Schnellerer Algorithmus spezifisch zur Hashfunktion als in \ref{bDayHash} zu finden?
  \item Brute-Force- und Wörterbuch-Angriff: Systematisches  Ausprobieren möglicher Eingaben und speicherung der Hashwerte in einer Lookup-Table.
  \item Time-/Memory-Tradeoff: Bilde Ketten von möglichen Eingaben. In der Rainbow-Table werden nur Start und Endwert der Kette gespeichert. Um Hashwert zu knacken müssen Teile der Kette durchlaufen werden.
\end{itemize}

 







