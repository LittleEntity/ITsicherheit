\chapter{Blockchiffren (6)}
Anwendung: Wenn Plaintext länger als vorgegebene Blocklänge des gewählten Chiffrierverfahrens. \\
Plaintext wird durch Padding\footnote{Padding engl. Auffüllen} auf eine Länge $k \cdot Blocklänge$ mit $k \in \mathbb{N}$ gebracht. Anschließend wird der Plaintext inklusive Padding in $k$ viele Teile zerlegt.
$$ P = (P_1, P_2, \ldots, P_k) $$

\section{Electronic Code Book (ECB) Mode}
Encryption: $E(P_i, K) = C_i$ \\
Decryption: $D(C_i, K) = P_i$ \\
Problem: Gleiche Plaintextblöcke ergeben gleiche Chiffretextblöcke! (besonders bei Bilddaten kann dies zu Enthüllungsproblemen führen)
\begin{figure}[htp]
	\centering
	\includegraphics[scale=1]{ECBblockchiffre.pdf}
	\caption{ECB Mode}
\end{figure}

\section{Cipher Block Chaining (CBC) Mode}
Encryption: $ C_0 = Initialisierungsvektor (IV); C_i = E(P_i \oplus C_{i-1}) $ \\
Decryption: $ P_i = C_{i-1} \oplus D(C_i, K) $ \\
Gleiche Klartextblöcke ergeben gleiche Chiffretextblöcke. (Bilddaten werden so ebenfalls gut verschlüsselt)
\begin{figure}[htp]
	\centering
	\includegraphics[scale=1]{CBCblockchiffre.pdf}
	\caption{CBC Mode}
\end{figure}





