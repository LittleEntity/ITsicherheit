%%%%%%%%%%%%%%%%%%%%%%%%%%%%%%%%%%%%%%%%%%%%%%%%%%%%%%
% Autor: Kajetan Weiß
% eMail: weissk@hochschule-trier.de
%
% Hinweise zur Verwendung und Ursprung:
% Dieses Dokument ist frei von Rechten Dritter. 
% Alle Abbildungen und Texte wurden von mir, Kajetan Weiß, verfasst. 
% Es ist auf Grundlage des der Vorlesung und Skript von Professor Gemmar der Hochschule Trier entstanden. 
% Wenn Du das Dokument für gut genug befindest würde es mich sehr freuen, wenn Du es anderen zugänglich machen würdest. 
% Zwei Bedingungen zur Verbreitung stelle ich: 
% Erstens, Vorwort und Nachwort müssen erhalten bleiben. 
% Zweitens, wenn Du Änderungen oder Ergänzungen vornimmst, bist Du herzlich dazu eingeladen dies zu tun, stelle aber an entsprechender Stelle oder am Anfang oder am Ende des Dokuments klar welche Änderungen oder Ergänzungen Du vorgenommen hast.
%%%%%%%%%%%%%%%%%%%%%%%%%%%%%%%%%%%%%%%%%%%%%%%%%%%%%%
%\documentclass[11pt,a4paper]{scrartcl}
%\documentclass[11pt,a4paper]{scrreprt}
\documentclass[11pt,a4paper,oneside]{scrbook}
%\documentclass[11pt,a4paper,oneside,draft]{scrbook}

\setlength{\parindent}{0em}
\setlength{\parskip}{1ex}

\usepackage[utf8]{inputenc}
\usepackage[T1]{fontenc}
\usepackage{lmodern}
\usepackage{ngerman}

\usepackage{amsmath} % align
\usepackage{amssymb} % für Mengensymbole \mathbb{<symbol>}; \checkmark

\usepackage[pdftex]{graphicx}
\graphicspath{{../Abbildungen_export/}, {../../Abbildungen_export/}}

\usepackage{wasysym}

\usepackage[usenames,dvipsnames,svgnames,table]{xcolor}
\definecolor{myLinkColor}{rgb}{0.25882353,0.36470588,0.60392157}

\usepackage[
    pdftex,
    a4paper,
    bookmarks,
    bookmarksopen=true,
    bookmarksnumbered=true,
    pdfauthor={Kajetan Weiß},
    pdftitle={Digitaltechnik Aufarbeitung},
    colorlinks,
    linkcolor=myLinkColor,
    urlcolor=myLinkColor
]{hyperref}

% usefull commands: \hspace{-\parindent}



\begin{document}

\title{IT-Sicherheit Zusammenfassung}
\author{Kajetan Weiß}
\date{\today}
\maketitle

\chapter*{Vorwort}
Schön, dass Du Dich entschieden hast mit dieser Arbeit zu lernen. Vorweg möchte ich raten nicht nur die Lektüre zu lesen sondern zum besseren Verständnis parallel zu jedem abgeschlossenen Kapitel die jeweiligen Übungsaufgaben zu erledigen. Wenn Du so vorgehst merkst du schnell ob Du alles richtig verstanden hast oder Du Dich noch intensiver mit dem Thema befassen musst. Manchmal kann es sich lohnen nach einem Abschnitt direkt mit den Übungsaufgaben zu beginnen und erst weiter zu lesen, wenn ein neuer Aufgabentyp eines Kapitels gestellt wird.

Außerdem möchte ich hier vorweg darauf hinweisen, dass ich keinerlei Garantien auf Korrektheit oder Vollständigkeit übernehme. Du kannst mir sehr gerne Fehler berichten oder mich auf Unvollständigkeiten hinweisen. Fehler, seien es inhaltliche, grammatikalische oder Rechtschreibfehler, werde ich umgehend korrigieren. Ergänzungen werde ich vornehmen sofern ich die Zeit dafür erübrigen kann.

%Nach Abschluss der Arbeit sind 14 TODOs offen geblieben. Im Quelltext sind die entsprechenden Stellen mit "`\%TODO"' markiert. Wenn Du interesse hast kannst Du Dich gerne um die TODOs kümmern und mir die Ergänzungen zuschicken.

Alle Abbildungen wurden von mir, Kajetan Weiß, verfasst. Das Dokument ist auf Grundlage der Vorlesung und Skript von Professor Knorr der Hochschule Trier entstanden. Die Textpassagen sind teilweise stark an das Skript von Professor Knorr angelehnt und teilweise aus seinem Skript zitiert. Ich habe mir erlaubt auf eine korrekte Zitierung zu verzichten um den Text lesbar zu halten und den Aufwand nicht übermäßig ansteigen zu lassen. Ich hoffe, dass dieser Hinweis ausreicht und möchte mich für hier für das legere Zitieren entschuldigen. Sollte sich jemand daran stören bitte ich um Kontaktaufname. Ich werde anschließend die Zitierung korrigieren.

Wenn Du das Dokument für gut genug befindest würde es mich sehr freuen, wenn Du es anderen zugänglich machen würdest. Zwei Bedingungen zur Verbreitung stelle ich: Erstens, Vorwort und Nachwort müssen erhalten bleiben. Zweitens, wenn Du Änderungen oder Ergänzungen vornimmst, bist Du herzlich dazu eingeladen dies zu tun, stelle bitte an entsprechender Stelle oder am Anfang oder am Ende des Dokuments klar welche Änderungen oder Ergänzungen Du vorgenommen hast.

Sollte irgendetwas unklar sein, kannst Du Dich gerne bei mir melden (siehe Kontaktdatentabelle \ref{kontakt}).
\begin{table}[htp]
\centering
\begin{tabular}{rl}
eMail & \texttt{k.weiss@hochschule-trier.de} \\
jabber/XMPP & \texttt{kajetan@jabber.ccc.de} \\ 
\end{tabular}
\caption{Kontaktdaten}
\label{kontakt}
\end{table}

Jetzt wünsche ich Dir viel Erfolg beim lernen, verstehen und lösen der Herausforderungen in der Veranstaltung.

\vspace{0.5cm}

\textsl{Kajetan Weiß, Trier den \today}

\tableofcontents

\chapter{Einführung}
Einführung einfügen\ldots

\part{Einführung in die Kryptologie}
\chapter{Einführung in die Kryptologie (VL2)}
\section*{Begriffe}
\begin{itemize}
  \item Kryptografie: Erstellung von kryptografischen Verfahren.
  \item Kryptoanalyse: Analyse kryptografischer Verfahren.
  \item Kryptologie: Sammelbegriff für Kryptografie \& Kryptoanalyse.
\end{itemize}

\section*{Schutzziele:}
\begin{itemize}
  \item Vertrauligkeit/Geheimhaltung
  \item Integrität (Schutz vor unentdeckter Manipulation)
  \item Authentifizierung/Authentizität
  \item Verbindlichkeit (Kausalkette zum Urheber)
  \item Anonymität/Pseudonymität
\end{itemize}

\section{Symmetrische Kryptosysteme (1.4)}
Key $K \in \{0,1\}^k$ \\
Plaintext $P \in \{0,1\}^*$ \\
Cyphertext $C \in \{0,1\}^*$ \\

encrypt:\\
\begin{math}
E: \{0.1\}^* \times \{0,1\}^k \mapsto \{0,1\}^* \\
E: (P, K) \mapsto C \\ 
\end{math}

decrypt:\\
$
D:  \{0,1\}^* \times \{0.1\}^k \mapsto \{0,1\}^* \\
D:  (C, K) \mapsto P \\ 
$

$$\forall_{K, P} \textnormal{ gilt } D_K(E_K(P)) = P$$

\begin{figure}[htp]
	\centering
	\includegraphics[scale=1]{KeyExchange.pdf}
	\label{KeyEx}
\end{figure}
Mit \emph{vertraulichem Austausch} von K können \emph{große Datenmengen} vertraulich übertragen werden. (1.6)

\section{Schlüssellänge vs Effektive Schlüssellänge (1.5)}
Gesamtlänge des Schlüssels $k$ \\
Effektive Länge des Schlüssels $k_e = ld(\textnormal{Anzahl verwendbarer Schlüssel})$

\section{Kerckhoffs'sche Prinzip (1.7)}
Sicherheit Kryptoverfahrens hängt ausschließlich von Geheimhaltung von K ab und nicht von Geheimhaltung des Kryptoverfahrens.

\section{Angriffsklassen (1.9)}
\begin{itemize}
  \item \emph{cyphertext only}: Suche nach $P$ oder $K$
  \item \emph{known plaintext}: Suche nach $K$
  \item \emph{chosen plaintext}: Suche nach $K$
  \item wenn Angriffe wiederholt angepasst stattfinden können: \emph{adaptive}
\end{itemize}

\section{One Time Pad (OTP) (2.3)}
Verschlüsselung jedes Plaintextbits $Pbit_i$ mit Schlüsselbit $Kbit_i$
\begin{align*}
E: Pbit_i \oplus Kbit_i \mapsto Cbit_i \\
D: Cbit_i \oplus Kbit_i \mapsto Pbit_i \\
\end{align*}
sicher wenn:
\begin{itemize}
  \item $K$ echt zufällig gewählt
  \item $Länge(K) \geq Länge(P)$
  \item Geheimnis $K$ 
\end{itemize}




\chapter{Moderne symmetrische Chiffren}
\section{Blockchiffre (3.1)}
definiert für: \\
feste Blocklänge $n$ für $P$ und $C$ \\
feste Schlüssellänge $k$ \\
Ein Block wird mit $K$ ver- bzw. entschlüsselt.

\section{Stromchiffre (vgl OTP) (3.2)}
unbegrenzte Folge von $Pbits$ und $Cbits$
Erzeugung einer unbegrenzten Folge von $Kbits$ aus $K$ mit Pseudozufallszahlengenerator $G$ (verwendet $K$ als Seed).
$$
G: \{0,1\}^k \mapsto \{0,1\}^*
$$
\begin{align*}
E: Pbit \oplus Gbit \mapsto Cbit \\
D: Cbit \oplus Gbit \mapsto Pbit \\
\end{align*}

\section{Feistel Cipher (3.3)}
Design-Prinzip für Block-Chiffren um beliebig lange $P$ zu verschlüsseln. \\
$n$-viele Runden R mit $n$-vielen Schlüsseln $K_i$ und Rundenfunktion $F$

\begin{figure}[htp]
	\centering
	\includegraphics[width=1\textwidth]{FeistelCipher.pdf}
\end{figure}


\subsection*{encrypt:}
\begin{enumerate}
  \item Teile Plaintext-Block $P$ in $L_0$ und $R_0$ auf (wobei $Länge(L_0) = Länge(R_0)$)
$$ P = L_0 \circ R_0 $$
  \item für die Iterationen $i$ gilt:

  \begin{tabular}{rl}
  	$i:$ & $ L_i = R_{i-1} $ \\
  	            &  $ R_i = L_{i-1} \oplus F( R_{i-1}, K_i ) $ \\ \hline
  	$i = 1:$ & $L_1 = R_0$ \\
  	            &  $R_1 = L_0 \oplus F(R_0, K_1)$ \\
  	$i = 2:$ & $L_2 = R_1$ \\
  	            &  $R_2 = L_1 \oplus F(R_1, K_2)$ \\
  	  & \vdots \\
  	 $i = n:$ & $ L_n = R_{n-1} $ \\
  	            &  $ R_n = L_{n-1} \oplus F( R_{n-1}, K_n ) $ \\
	\end{tabular}
\end{enumerate}

\subsection*{decrypt:}
\begin{enumerate}
  \item Teile Cyphertext-Block $C$ in $L_n$ und $R_n$ auf (wobei $Länge(L_n) = Länge(R_n)$)
$$ C = L_n \circ R_n $$
  \item für die Iterationen $i$ gilt:

  \begin{tabular}{rl}
  	$i:$ &  $ R_{i-1} = L_i $ \\
  	            & $ L_{i-1} = R_{i} \oplus F( L_{i}, K_i ) $ \\ \hline
  	$i = n:$ &  $ R_{n-1} = L_n $ \\
  	            & $ L_{n-1} = R_{n} \oplus F( L_{n}, K_n ) $ \\
  	  & \vdots \\
  	$i = 1:$ &  $ R_{1} = L_2 $ \\
  	            & $ L_{1} = R_{2} \oplus F( L_{2}, K_2 ) $ \\
  	$i = 0:$ &  $ R_{0} = L_1 $ \\
  	            & $ L_{0} = R_{1} \oplus F( L_{1}, K_1 ) $ \\
	\end{tabular}
\end{enumerate}

funktioniert für jedes F. Verfahren ist nur sicher mit bestimmten F.

\section{Data Encryption Standard (DES) (3.4)}
\begin{itemize}
  \item Blocklänge: 64bit
  \item Runden: 16
  \item Sclüssellänge: 64bit, davon 8 Paritybits, also effektive Schlüssellänge 56bit
  \item Sicherheit durch S-Boxen (Substitutionsboxen) für $F$ als Rundenfunktion
  \item TripleDES: 3 fache Anwendung von DES \\
  	$2 K$: 112bit: $C=E(K_1,D(K_2,E(K_1,P)))$ \\
    $3 K$: 112bit: $C=E(K_1,D(K_2,E(K_3,P)))$ \\  
\end{itemize}

\section{Advanced Encryption Standard (AES) (3.5)}
\begin{itemize}
  \item Blocklänge: 128bit
	\item Runden: 10 bei 128bit; 12 bei 192bit; 14 bei 256bit
  \item Schlüssellänge: 128bit, 192bit oder 256bit
  \item Geringer Implementierungsaufwand in HW und SW
  \item einfaches Algorithmen-Design
  \item lizenzkostenfrei
\end{itemize}










\chapter{Hashfunktionen}
\section{Anwendungen (4.1)}
\begin{itemize}
  \item Integritätscheck
  \item sichere Komprimierung von z.B. digitalen Signaturen
\end{itemize}

\section{Einwegfunktion (4.2)}
Für Einwegfunktion $F: X \mapsto Y$ gilt:
\begin{itemize}
  \item $\forall_{x \in X} : F(x) \textnormal{ leicht berechenbar}$
  \item $\forall_{y \in Y}$: es ist praktisch unmöglich ein Urbild $x \in X$ zu berechnen für das $F(x) = y $ ist.
\end{itemize}

\section{Hashfunktionen (4.3)}
Für Hashfunktion $H: \{0,1\}^* \mapsto \{0,1\}^n$
\begin{itemize}
  \item H ist Einwegfunktion
  \item Sei $x \in \{0,1\}^*$ . Es ist praktisch unmöglich ein $z \neq x$ zu berechnen, sodass $H(z) = H(x)$ gilt.
  \item Kollisionsresistenz: praktisch unmöglich frei wählbares Paar $(x,z)$ zu berechnen, sodass $H(x) = H(z)$.
\end{itemize}
Ohne die letzte Bedingung wird die Funktion schwache Hashfunktion genannt.

Umgangssprachlich: Hashfunktion ist eine kryptografisch sichere, nicht umkehrbare Komprimierungsfunktion.

\section{Bekannte Hashfunktionen (4.4)}
\begin{itemize}
  \item MD2, MD4, MD5: 128bit
  \item SHA-1: 160bit
  \item SHA-2: 224bit, 256bit, 384bit, 512bit
  \item SHA-3: 224bit bis 512bit
\end{itemize}

(4.5) SHA-1 im Detail VL2 S.12

\section{Merkle-Damgard Konstruktion von Hashfunktionen (4.6)}
Gegeben: Einwegfunktion $F: \{0,1\}^{m+n} \mapsto \{0,1\}^n$ und Initialisierungsvektor $H_0$\\
Konstruiere $F: \{0,1\}^{*} \mapsto \{0,1\}^n$
\begin{enumerate}
  \item Erweitere $P$, sodass $Länge(P)$ ein Vielfaches von $m$ ist. Das Ergebnis sei $P_m$.
  \item Teile $P_m$ in Blöcke $p_1, p_2, \ldots, p_k$ mit der Länge $m$.
  \item Berechne $H(P_m)$ mit Iterationen über die Blöcke aus 2. und der Funktion $F$ wie folgt:
  	\begin{align*}
  		H_i &= F(y_1 \circ H_0) \\
  		H_2 &= F(y_2 \circ H_1) \\
  		& \vdots \\
  		H_k &= F(y_k \circ H_{k-1}) = H(P_m) \\
		\end{align*} 
\end{enumerate}

\section{Birthday Paradox (4.7)}
Wahrscheinlichkeit P(z,N), dass mit z vielen unabhängigen Zügen aus Wertemenge \mathbb{N} ein Wert doppelt vorkommt, ist ab $z = 1,17 \cdot \sqrt{N}$ größer als $\frac{1}{2}$. 

Beispiel: Wahrscheinlichkeit dass zwei Menschen am gleichen Tag Geburtstag haben ist ab 23 Menschen größer als $\frac{1}{2}$.

\section{Anwendung des Birthday Paradox auf Hashfunktionen (4.8)}
\label{bDayHash}
Gesucht ist eine beliebige Kollision.\\
Algorithmus:
\begin{enumerate}
  \item Wähle zufällig ein bisher noch nicht betrachtetes $x_i$.
  \item Vergleiche $H(x_i)$ mit allen anderen bisher betrachteten $H(x_j)$.
  \item Wenn keine Kollision gefunden wurde, beginne wieder bei 1.
\end{enumerate} 
$H(x_i) = H(x_j)$ ergibt sich im mittel nach $ 1.17 \sqrt{2^n} $ Iterationen.

\section{Angriffe auf Hashfunktionen (4.9)}
\begin{itemize}
  \item Kollisionsangriff: Schnellerer Algorithmus spezifisch zur Hashfunktion als in \ref{bDayHash} zu finden?
  \item Brute-Force- und Wörterbuch-Angriff: Systematisches  Ausprobieren möglicher Eingaben und speicherung der Hashwerte in einer Lookup-Table.
  \item Time-/Memory-Tradeoff: Bilde Ketten von möglichen Eingaben. In der Rainbow-Table werden nur Start und Endwert der Kette gespeichert. Um Hashwert zu knacken müssen Teile der Kette durchlaufen werden.
\end{itemize}

 








\chapter{Asymmetrische Kryptographie / Public Key Verfahren}
\section{Definition (5.2)}
Public Key $pK$ \\
Secret Key $sK$ \\
encrypt: $E: F(P, pK) \mapsto C $ \\
decrypt: $D: F(C, sK) \mapsto P $ \\

\begin{figure}[htp]
	\centering
	\includegraphics[scale=1]{AsymKrypto.pdf}
\end{figure}

\section{Mathematische Aufgaben auf denen asymmetrische Kryptoverfahren basieren (5.3)}

\subsection{Faktorisierung (5.3.1)}
Maschinell das Produkt zweier Primzahlen zu berechnen ist schnell. \\
Dagegen ist kein Algorithmus mit polynomieller Bekannt, welcher eine gegebene Zahl in ihre Primfaktoren zerlegt. \\
Laufzeiten von bekannten Algorithmen zur Primfaktorzerlegung einer Zahl mit $n$-bit:
\begin{itemize}
  \item Probedivision $O(2^\frac{n}{2})$
  \item Quadratisches Sieb $O(exp(n^\frac{1}{2}(ln(n))^\frac{1}{2})$
  \item Zahlenkörpersieb $O(exp(n^\frac{1}{3}(ln(n))^\frac{2}{3})$
\end{itemize}

\subsection{Diskreter Logarithmus (5.3.2)}
\begin{align*}
a &= g^e \hspace{4pt} mod \hspace{4pt} p \\
e &= log_g(a) \hspace{4pt} mod \hspace{4pt} p \\
\end{align*}
Es existieren Algorithmen $\in O(n^c)$ für ein gegebenes $e$ um die Gleichung $a = g^e \hspace{4pt} mod \hspace{4pt} p$ zu lösen.

Es sind bisher \emph{keine Algorithmen} $\in O(n^c)$ bekannt welche für gegebene $a,g,p$ die Gleichung $e = log_g(a) \hspace{4pt} mod \hspace{4pt} p$ lösen.

Aus Erfahrung wird gesagt, dass das Lösen von Diskreten-Algorithmen-Gleichungen nicht schwerer ist als das Lösen einer Faktorisierungs-Gleichung.


\subsection{Quadratwurzel ziehen mod n (5.3.3) ??????}++
"`Für n=p*q mit p und q Primzahlen gibt es ohne Kenntnis von p und q kein effizientes Verfahren zur Bestimmung der Quadratwurzeln.\\
Die Äquivalenz zum Faktorisierungsproblem kann bewiesen werden."'

\subsection{Überblick über ausgewählte Asymmetrische Verfahren (5.3.4)}
\begin{tabular}{*{5}{l}}
Verfahren & schweres Problem / Falltür & leichtes Problem & Signatur möglich & Einsatz in der Praxis \\
RSA & Faktorisierung mod n & Multiplizieren/Potenzieren mod n & ja & ja \\
Rabin & Quadratwurzel ziehen mod n & Quadrieren mod n & nein & nein \\
Elgamal & Diskreter Logarithmus mod n & Multiplizieren/Potenzieren mod n & ja & ja \\
\end{tabular}

\section{Mathematische Vorbereitung für RSA (5.4)}
\subsection{Division mit Rest}
$$
\textnormal{ganze Zahl } z \in \mathbb{Z}
$$
$$
 \textnormal{Modul } m \in \mathbb{N}
$$
$$
z = qb + r \textnormal{ mit } qb = \lfloor z : m \rfloor \textnormal{ und } r = z - qb = z \hspace{4pt} mod \hspace{4pt} m 
$$

\subsection{Modulares Rechnen}
Es gelten die gewohnten arithmetischen Rechengesetze. Jedes Teilergebnis lässt sich zusätzlich um $ mod \hspace{4pt} m$ reduzieren.

\begin{align*}
((x \hspace{4pt} mod \hspace{4pt} m) + (y \hspace{4pt} mod \hspace{4pt} m)) \hspace{4pt} mod \hspace{4pt}  m =& (x+y) \hspace{4pt} mod \hspace{4pt} m \\
((x \hspace{4pt} mod \hspace{4pt} m) \cdot (y \hspace{4pt} mod \hspace{4pt} m)) \hspace{4pt} mod \hspace{4pt}  m =& (x \cdot y) \hspace{4pt} mod \hspace{4pt} m \\
((z \hspace{4pt} mod \hspace{4pt} m) \cdot ((x \hspace{4pt} mod \hspace{4pt} m) + (y \hspace{4pt} mod \hspace{4pt} m))) \hspace{4pt} mod \hspace{4pt} m =& ((z \hspace{4pt} mod \hspace{4pt} m) \cdot (x \hspace{4pt} mod \hspace{4pt} m) \\
&+ (z \hspace{4pt} mod \hspace{4pt} m) \cdot (y \hspace{4pt} mod \hspace{4pt} m)) \hspace{4pt} mod \hspace{4pt} m \\ 
\end{align*}

\subsection{Wiederholtes Quadrieren (5.4.3)}
um $b^e \hspace{4pt} mod \hspace{4pt} n$ zu berechnen\footnote{http://de.wikipedia.org/wiki/Bin\%C3\%A4re\_Exponentiation}:
\begin{itemize}
  \item Schreibe Exponenten $e$ als Dualzahl
  \item Starte mit Zwischenergebnis $z = 1$
  \item Arbeite die Stellen vom höchstwertigsten zum niederwertigsten Bitstelle ab.
  	\begin{itemize}
  		\item wenn Stelle 0: $z := z^2$
  		\item wenn Stelle 1: $z := z^2 \cdot b$
		\end{itemize}
\end{itemize}

$$\textnormal{Beispiel: } 3^{13} \hspace{4pt} mod \hspace{4pt} 17$$
$$13_{10} = 1101_2$$
$$
\begin{array}{crclc}
	1 & 1^2 \cdot 3 & = & 3 &  (mod \hspace{4pt} 17) \\
	1 & 3^2 \cdot 3 & = & 10 &  (mod \hspace{4pt} 17) \\
	0 & 10^2 & = & -2 &  (mod \hspace{4pt} 17) \\
	1 & (-2)^2 \cdot 3 & = & 12 &  (mod \hspace{4pt} 17) \\
\end{array}
$$

\subsection{Multiplikatives Inverses (5.4.4)}
Das Multiplikatives Inverse $z^{-1}$ zu $z$ ist definiert über:
$$z \cdot z^{-1} = 1 \hspace{4pt} (mod \hspace{4pt} m) $$

\subsection{Definition: Primzahl (5.4.5)}
$p$ ist Primzahl, wenn $p$ außer sich selbst und 1 keine positiven Teiler hat.

\subsection{größter gemeinsamer Teiler (ggT) (5.4.6)}
Der ggT zweier Zahlen ist die größte Zahl, welche beide Zahlen ohne Rest teilt. \\
Falls $ggT(a,b) = 1$, so heißen a und b teilerfremd.

\subsection{Erweiterter Euklidischer Algorithmus zur Bestimmung des Multiplikativen Inversen (5.4.7)}
Wenn eine Zahl $z$ zum Modul $m$ teilerfremd ist, lässt sich ihr Multiplikatives Inverses $z^{-1}$ mit dem Erweiterten Euklidischen Algorithmus bestimmen.\footnote{genaue Erläuterung: "`http://www.yimin-ge.com/doc/multiplikative\_inverse.pdf"'} Der Algorithmus liefert auch Aufschluss über die Teilerfremdheit.
\begin{enumerate}
  \item starte mit erster Gleichung der Form $m = 1 \cdot m + 0 \cdot z$ und nachfolgender Gleichung der Form $z = 0 \cdot m + 1 \cdot z$
  \item Bilde die Folgegleichungen indem von der vorletzten Gleichung ein vielfaches der letzten Gleichung subtrahiert wird, sodass das Ergebnis der linken Seite dem Rest der Division mit Rest entspricht.
  \item Wiederhole 2. solange bis das Ergebnis der linken Seite 1 oder 0 ist.
  \begin{itemize}
    \item Wenn 1: Das Multiplikative Inverse $z^{-1}$ ist der Faktor vor $z$ .
    \item Wenn 0: Es gibt kein Multiplikatives Inverses von $z$ zum Modul $m$, da $z$ und $m$ nicht teilerfremd sind.
	\end{itemize}
\end{enumerate}
Beispiel: Finde $z^{-1}$ zu $7 \hspace{4pt} (mod \hspace{4pt} 23) $:
$$
\begin{array}{crcll}
I & 23 & = & 1 \cdot 23 + 0 \cdot 7 & \\
II & 7 & = & 0 \cdot 23 + 1 \cdot 7 & \\
III & 2 & = & 1 \cdot 23 + (-3) \cdot 7 & I + II \cdot (-3) \\
IV& 1 & = & (-3) \cdot 23 + 10 \cdot 7 & II + III \cdot (-3) \\
\end{array}
$$
Ergebnis: Das Multiplikative Inverse zu 7 zum Modul 23 ist 10.
\begin{center}
\begin{tabular}{*{4}{r}}
23 & 1 & 0 & \\
7 & 0 & 1 & -3\\
2 & 1 & -3 & -3\\
1 & -3 & \fbox{10} & \\
\end{tabular}
\end{center}

\subsection{Eulersche $\varphi$ Funktion (5.4.8)}
$\varphi$(n) = Anzahl der positiven ganzen Zahlen $z \le n$, mit $ggT(z,n) = 1$ . \\
Beispiel: $\varphi(21) = 12$ \\
speziell für Primzahlen $p, q$ : \\
$\varphi(p) = p-1$ \\
$\varphi(p \cdot q) = (p-1) \cdot (q-1)$ wobei $p \neq q$ \\

\subsection{Satz von Euler (5.4.9)}
Falls $ggT(x, n) = 1$, dann gilt  $x^{\varphi(n)} = 1 \hspace{4pt} (mod \hspace{4pt} n) $ 

\section{RSA-Verfahren (5.5)}
\subsection{Definition des RSA-Verfahrens}
Der Empfänger bildet \textsc{rsa}-Modul $n$ mit zwei Primzahlen $p$ und $q$:

$$
	n = p \cdot q
$$

Öffentliche Exponent $e$ (encrypt) wird gebildet mit:

$$
	1 < e < (p-1)(q-1) \wedge ggT(e,(p-1)(q-1)) = 1
$$

Öffentlicher Schlüssel ist das Paar $(n,e)$.

Geheime Exponent $d$ (decrypt) wird gebildet mit:

$$
	1 < d < (p-1)(q-1) \wedge e \cdot d \hspace{4pt} mod \hspace{3pt} (p-1)(q-1) = 1
$$

Geheimer Schlüssel ist Tripel $(p,q,d)$.

Hinweis: Die Berechnung von $d$ erolgt $mod \hspace{4pt} \varphi(n)$.

\subsection{RSA-Verschlüsselung und Entschlüsselung}
Encryption of Plaintext $P$ with $0 \le P < n$:
$$ C = (P^e) \hspace{4pt} (mod \hspace{4pt} n)  $$

Decryption of Cyphertext $C$:
$$ P = (C^d) \hspace{4pt} (mod \hspace{4pt} n)  $$

RSA Schlüsselänge ist Anzahl Bits des Moduls $n$.

$ P \mapsto P^e \hspace{4pt} (mod \hspace{4pt} n) $ ist eine Falltürfunktion da:
Ohne geheimem Schlüssel ist die Funktion praktisch nicht invertierbar. \\
Mit Kenntnis des geheimen Schlüssel ist die Invertierung einfach.

\subsection{Falltürfunktion (trap door function)}
 Für ein Geheimnis $S$ ist Funktion $F_S: X \mapsto Y$ eine Falltürfunktion, wenn gilt:
 \begin{enumerate}
  \item $\forall x \in X$ ist $F(x)$ leicht berechenbar.
  \item Ohne Kenntnis des Geheimnisses $S$ ist es für gegebenes $y \in Y$ praktisch nicht möglich ein Urbild $x$ von $y$ unter $F_S$ zu berechnen.
  \item Mit Kenntnis des Geheimnisses $S$ ist es für gegebenes $y \in Y$ leicht ein Urbild $x$ von $y$ unter $F_S$ zu berechnen. 
\end{enumerate}

\subsection{Sicherheit des RSA}
\begin{itemize}
  \item Aufwand um geheimen Exponenten aus öffentlichem Schlüssel zu berechnen ist äquivalent zur Faktorisierung des RSA-Moduls
  \item Bislang unbekannt RSA brechbar ohne Modul zu faktorisieren oder geheimen Exponenten $d$ zu finden.
  \item Der Aufwand zur Faktorisierung des \textsc{rsa}-Moduls hängt von dessen Größe ab.
\end{itemize}

\subsection{Sichere Auswahl von RSA Parametern}
\begin{itemize}
  \item RSA-Schlüsssellänge $\geq$ 1024 Bit.
  \item Primzahlen $p$ und $q$ sollten zufällig, gleichverteilt gewählt sein.
  \item wegen "`low exponent attack"' sollte öffentlicher Exponent $e$ nicht zu klein sein: oft wird $e = 2^{16}+1 = 65537$ gewählt.
\end{itemize}








\chapter{Blockchiffren (6)}
Anwendung: Wenn Plaintext länger als vorgegebene Blocklänge des gewählten Chiffrierverfahrens. \\
Plaintext wird durch Padding\footnote{Padding engl. Auffüllen} auf eine Länge $k \cdot Blocklänge$ mit $k \in \mathbb{N}$ gebracht. Anschließend wird der Plaintext inklusive Padding in $k$ viele Teile zerlegt.
$$ P = (P_1, P_2, \ldots, P_k) $$

\section{Electronic Code Book (ECB) Mode}
Encryption: $E(P_i, K) = C_i$ \\
Decryption: $D(C_i, K) = P_i$ \\
Problem: Gleiche Plaintextblöcke ergeben gleiche Chiffretextblöcke! (besonders bei Bilddaten kann dies zu Enthüllungsproblemen führen)
\begin{figure}[htp]
	\centering
	\includegraphics[scale=1]{ECBblockchiffre.pdf}
	\caption{ECB Mode}
\end{figure}

\section{Cipher Block Chaining (CBC) Mode}
Encryption: $ C_0 = Initialisierungsvektor (IV); C_i = E(P_i \oplus C_{i-1}) $ \\
Decryption: $ P_i = C_{i-1} \oplus D(C_i, K) $ \\
Gleiche Klartextblöcke ergeben gleiche Chiffretextblöcke. (Bilddaten werden so ebenfalls gut verschlüsselt)
\begin{figure}[htp]
	\centering
	\includegraphics[scale=1]{CBCblockchiffre.pdf}
	\caption{CBC Mode}
\end{figure}






\chapter{Hybride Verschlüsselung (7)}
\section{Vergleich von asymmetrischen und symmetrischen Verfahren}
\begin{itemize}
  \item \emph{Schlüsselaustausch}: Symmetrische Kryptosysteme erfordern einen vertraulichen Schlüsselaustausch. Bei Asymmetrischen Verfahren ist dies nicht notwendig.
  \item \emph{Schlüsselanzahl im Netzwerk}: Symmetrische Kryptosysteme benötigen für jedes Kommunikationspaar einen eigenen geheimen Schlüssel (bei n Kommunikationspartnern n(n-1)/2 viele Schlüssel). Bei asymmetrischen Systemen hat jeder Teilnehmer einen öffentlichen und einen privaten Schlüssel (2n viele Schlüssel).
  \item \emph{Performanz}: Asymmetrische Verfahren arbeiten mit komplexen mathematischen Operationen und die Berechnung zur Ver- bzw. Entschlüsselung ist im Vergleich zu symmetrischen Verfahren sehr langsam.
\end{itemize}

\section{Definition Hybride Verschlüsselung (7.2)}
Kombination symmetrischer und asymmetrischer Kryptosysteme:
\begin{itemize}
  \item Nutzung eines asymmetrischen Kryptosystems zum geheimen Schlüsselaustausch eines symmetrischen \emph{Session Keys}.
  \item Nutzung eines symmetrischen Kryptosystems zur Verschlüsselung der Nachricht mit dem \emph{Session Key}.
\end{itemize}
\subsection*{Ablauf:}
\begin{enumerate}
  \item Sender wählt (pseudo-)zufällig einen Session Key $K$.
  \item Sender verschlüsselt $K$ asymmetrisch mit dem Public Key $PK$ des Empfängers: $PE(PK, K)$
  \item Sender sendet $PE(PK, K)$ an den Empfänger.
  \item Sender und Empfänger können mit symmetrischem Verschlüsselungsverfahren über $K$ geschützt kommunizieren. 
\end{enumerate}
\begin{figure}[htp]
	\centering
	\includegraphics[width=.8\textwidth]{HybridCrypt.pdf}
	\caption{Hybrid Crypto System}
\end{figure}

\chapter{Nachrichtenauthentisierung (8)}
Es wird unterschieden:
\begin{itemize}
  \item Integrität: Nachweis, dass empfangene Nachricht gleich der ursprünglich gesendeten Nachricht ist. Wird mittels kryptografischer Prüfsummen realisiert ("`Message Authentication Codes"' krzl. MAC).
  \item Senderauthentizität: digitale Unterschrift (Digitale Signaturen mittels asymmetrischer Kryptoverfahren)
\end{itemize}

\section{Kryptografische Prüfsummen (MAC) (8.1)}
\begin{itemize}
  \item $mac$ Funktion berechnet kryptografische Prüfsumme einer Nachricht $M$ mit Schlüssel $K$. 
  	$$ mac: \{0,1\}^k \times \{0,1\}^* \mapsto \{0,1\}^n $$
	\item Sender sendet Nachricht $M$ und $mac(K,M)$.
	\item Empfänger weiß nach Prüfung des MAC, dass Nachricht unverändert ist und vom angegebenen Sender stammt.
	\item Sicherheitsanforderungen
	\begin{itemize}
  	\item Ohne Kenntnis von $K$ soll es praktisch niemandem möglich sein, für eine Nachricht M eine gültige Prüfsumme zu berechenen. 
  	\item Bei gegebener Nachricht $M$ und Prüfsumme $mac(K,M)$ darf es einem Angreifer praktisch weder möglich sein $K$ zu berechnen noch eine weitere Nachricht $M'$ mit identischer Prüfsumme zu finden. 
	\end{itemize}
\end{itemize}

\begin{figure}[htp]
	\centering
	\includegraphics[width = 0.8\textwidth]{MsgAuthCodes.pdf}
	\caption{symmetric message authentication codes}
\end{figure}


\subsection* {Abgrenzung zu Fehler erkennenden Codes}
Fehler erkennende Codes dienen dazu zufällige Übertragungsfehler zu erkennen. Sie schützen nicht vor Manipulation, da gültige Codes zu gegebener Nachricht leicht neu berechnet werden können. Fehler erkennende Codes dienen also ausschließlich der Safety und \emph{nicht} der Security.

\section{Verfahren zur MAC Berechnung}
\subsection{Blockchiffren im CBC Modus}
Da der letzte Chiffretextblock $C_n$ bei der CBC-Verschlüsselung von allen Klartextblöcken abhängt, kann dieser zur Integritätsprüfung und Authentisierung verwendet werden. Protokollablauf CBC-MAC:
\begin{enumerate}
  \item Sender sendet Nachricht $M$ und als MAC den letzten Chriffretextblock $C_n$ der CBC Verschlüsselung von $M$ mit $E_K$.
  \item Empfänger berechnet selbst $C_n'$ mittels CBC Verschlüsselung von empfangener Nachricht $M'$ mit $E_K$. Ist $C_n' = C_n$ so ist davon auszugehen, dass die Nachricht unverfälscht und vom angegebenen Empfänger übertragen wurde. (Da nur Empfänger und Sender den Schlüssel $K$ zur Verschlüsselung kennen). 
\end{enumerate}

\subsection{Einsatz von Hashfunktionen}
Protokollablauf:
\begin{enumerate}
  \item Sender sendet Die Nachricht M und als MAC $mac_M = hash(concatenate(K, M))$
  \item Empfänger empfängt Nachricht $M'$ und berechnet $mac_M' = hash(concatenate(K, M'))$. Akzeptiert wenn $mac_M' = mac_M$ .
\end{enumerate}
Verfahren ist grundsätzlich sicher, weist allerdings bei üblichen Hashfunktionen Schwächen auf.

\subsubsection*{Hashed MAC Standard (HMAC)}
MAC Funktion mit Paddings $p_1$ und $p_2$ um Schlüssel $K$ auf Länge eines Blocks der Hashfunktion aufzufüllen.
  	$$hmac(K,M) = hash((K \oplus p_1) \mid hash(K \oplus p_2 \mid M))$$
(in Standard RFC 2104 definiert)

\section{Verbleibende Schwachstellen symmetrischer Verfahren (8.4)}
\begin{itemize}
  \item Bisher betrachtete Nachrichtenauthentisierungen schützen nicht vor Replay-Angriffen. Falls Schutz erforderlich: Einbinden eines Zeitstempels in die Nachricht.
  \item Der Sender kann behaupten für eine Nachricht mit korrekter Prüfsumme, diese nicht gesendet zu haben. (Der Empfänger kann zu jeder beliebigen Nachricht einen gültigen MAC erzeugen.) 
\end{itemize}
\chapter{Digitale Signaturen}
\section{Anforderungen an Digitale Signaturen (9.1)}
Eine Digitale Signatur muss:
\begin{itemize}
  \item vom Nachrichteninhalt abhängen.
  \item den Absender eindeutig ausweisen.
  \item einfach zu erzeugen und zu prüfen sein.
  \item fälschungssicher sein: Der Empfänger darf keine gültige Signatur vom Sender erzeugen können. Damit kann der Sender nicht mehr leugnen die Signatur angefertigt zu haben.
  \item einfach gespeichert werden können.
  \item über einen gewissen Zeitraum beweiskräftig bleiben.
\end{itemize}

\section{Konstruktion von Digitalen Signaturen (9.2)}
Digitale Signaturen können mit asymmetrischen Kryptosystemen realisiert werden. Gegeben sei eine kryptografisch sichere Hashfunktion $H$, ein asymmetrisches Kryptosystem $(E,D)$ und dem Schlüsselpaar $(PK,SK)$. Die Entschlüsselungsfunktion $D$ wird als Signaturfunktion $sign$ benutzt. Gültige Signaturen der Nachricht $M$ des Senders sind $sign(SK, M)$ und $sign(SK, H(M))$.

Zur Prüfung wird die Verschlüsselungsfunktion $E$ als Prüffunktion $validate$ genutzt. Eine Prüfung ist genau dann erfolgreich wenn gilt:
$$
	validate(PK, sign(M)) = M
$$
beziehungsweise:
$$
	validate(PK, sign(H(M))) = H(M)
$$

Dieses Verfahren ist kryptografisch sicher, solange der Secret Key $SK$ des Senders nur dem Sender bekannt ist.

\begin{figure}[htp]
	\centering
	\includegraphics[width = 0.8\textwidth]{DigitalSignature.pdf}
	\caption{Signing of message $M$ with sender's secret key $SK$ and validation by receiver with sender's public key $PK$}
\end{figure}




%\chapter{Nachwort}
Vielen Dank dafür, dass Du Dir die Zeit nimmst die paar Zeilen des Nachwortes zu lesen. Ich hoffe du hattest Erfolg beim Lernen und Verstehen der Kapitel.

Ich habe dieses Dokument in Eigeninitiative verfasst und hoffe, dass außer mir auch viele andere mit diesem Dokument lernen können. Die gesamte Zusammenstellung hat ungefähr einen Monat gedauert. Wenn dieses Dokument Dir weiterhelfen konnte und Du mir gerne ein Bier oder eine Pizza dafür ausgeben möchtest, kannst Du das sehr gerne tun, indem Du per PayPal auf die Addresse
\begin{center}
	\texttt{weissk@hochschule-trier.de}
\end{center}
einen Betrag Deiner Wahl spendest. Ich bin ein armer Student, wie Du wahrscheinlich auch und freue mich übere jede kleine Zuwendung. Es lebe das Crowd-Funding! \smiley  

Dieses Dokument ist, wie ihr dem Vorwort entnehmen könnt open-source. Bitte unterstütze open-source Projekte und spende für Projekte, die Dir gefallen oder beteilige Dich  am besten an einem oder stelle selbst eines zur Verfügung.

\vspace{2cm}

\textsl{Kajetan Weiß, Trier den \today}
\end{document}